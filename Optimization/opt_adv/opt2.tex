\def\mytitle{Optimization Assignment-2}
\def\myauthor{G.Kumar}
\def\contact{kumargandhamaneni20016@gmail.com}
\def\mymodule{Future Wireless Communication (FWC)}
\documentclass[10pt, a4paper]{article}
\usepackage[a4paper,outer=1.5cm,inner=1.5cm,top=1.75cm,bottom=1.5cm]{geometry}
\twocolumn
\usepackage{graphicx}
\graphicspath{{./images/}}
\usepackage[colorlinks,linkcolor={black},citecolor={blue!80!black},urlcolor={blue!80!black}]{hyperref}
\usepackage[parfill]{parskip}
\usepackage{lmodern}
\usepackage{tikz}
\usepackage{physics}
\usepackage{tabularx}
\usetikzlibrary{calc}
\usepackage{amsmath}
\usepackage{amssymb}
\renewcommand*\familydefault{\sfdefault}
\usepackage{watermark}
\usepackage{lipsum}
\usepackage{xcolor}
\usepackage{listings}
\usepackage{float}
\usepackage{titlesec}
\providecommand{\mtx}[1]{\mathbf{#1}}
\titlespacing{\subsection}{1pt}{\parskip}{3pt}
\titlespacing{\subsubsection}{0pt}{\parskip}{-\parskip}
\titlespacing{\paragraph}{0pt}{\parskip}{\parskip}
\newcommand{\figuremacro}[5]{
    \begin{figure}[#1]
        \centering
        \includegraphics[width=#5\columnwidth]{#2}
        \caption[#3]{\textbf{#3}#4}
        \label{fig:#2}
    \end{figure}
}
\newcommand{\myvec}[1]{\ensuremath{\begin{pmatrix}#1\end{pmatrix}}}
\let\vec\mathbf
\lstset{
frame=single, 
breaklines=true,
columns=fullflexible
}
\thiswatermark{\centering \put(0,-105.0){\includegraphics[scale=0.35]{iith}} }
\title{\mytitle}
\author{\myauthor\hspace{1em}\\\contact\\IITH\hspace{0.5em}-\hspace{0.5em}\mymodule}
\date{}
\begin{document}
	\maketitle
\paragraph{\textit{Problem Statement}}- Prove that the least perimeter of an isosceles triangle in which a circle of radius r can be inscribed is $6\sqrt{3}r$.

\section*{\large Solution}
\begin{figure}[H]
\centering
\includegraphics[width=1\columnwidth]{opt2.png}
\caption{Graph of f(x)}
\label{fig:triangle}
\end{figure}

Let x be the length of each of the two equal sides, y be the length of the third side of the $\triangle{ABC}$ and $\angle{B}=\angle{C}$.\\
Then,
\begin{align}
x=r\cot{\frac{C}{2}}+r\cot{\frac{A}{2}}\hspace{1em}and\hspace{1em}y=2r\cot{\frac{C}{2}}
\end{align}	
Now, perimeter of $\triangle{ABC}$ is,
\begin{align}
P=2x+y
\end{align}
\begin{align}
\implies P=2r[2\cot{\frac{C}{2}}+\cot{\frac{A}{2}}]
\end{align}
Since, $\frac{A}{2}=\frac{\pi}{2}-C$,
\begin{align}
\implies P=2r[2\cot{\frac{C}{2}}+\tan{C}]
\end{align}
    \subsection*{\normalsize Gradient descent}
    
    
    \begin{align}
	\label{eq:vol_varx}
	f(x) =2r[2\cot{\frac{x}{2}}+\tan{x}]\\
    f'(x) = 2r[-cosec^2{\frac{x}{2}}+\sec^2{x}]
	\end{align}

we have to attain the maximum value of f(x). This can be seen in Figure f(x).Using gradient descent method we can find its minima value.
\begin{equation}
        x_{n+1} = x_n - \alpha \nabla f(x_n) \\
\end{equation}
\vspace{1mm}
\begin{equation}
\implies x_{n+1}=x_n+\alpha(2r[-cosec^2{\frac{x}{2}}+\sec^2{x}])
\end{equation}

Taking $x_0=0.5,\alpha=0.001$ and precision = 0.00000001, values obtained using python are:
    

    \begin{align}
        \text{Minima} = 10.3923r\\
        \implies \boxed{\text{Minima} = 6\sqrt{3}r}\\
        \boxed{\text{Minima Point} = 1.0471}
    \end{align}
    \begin{align*}
    \therefore  Hence\hspace{0.5em}Proved 
    \end{align*} 
   \end{document}